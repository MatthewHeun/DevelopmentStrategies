\documentclass{article}

\usepackage{url}

\begin{document}

\title{Git Branching Strategy}
\author{Matthew Kuperus Heun}

\maketitle


%%%%%%%%%%%%%%%%%%%%%%%%%%%%%%%%%%%%%%%%%%%%%%%%%%%%%%%%%%%%%%
\begin{abstract}
%%%%%%%%%%%%%%%%%%%%%%%%%%%%%%%%%%%%%%%%%%%%%%%%%%%%%%%%%%%%%%

This document gives my strategy for branching in a git repository. 
The strategy is based on the fine blog post by Vincent Driessen at\\
\url{http://nvie.com/posts/a-successful-git-branching-model/}
and other resources.
Adherence to this strategy will enable better development processes
\end{abstract}


%%%%%%%%%%%%%%%%%%%%%%%%%%%%%%%%%%%%%%%%%%%%%%%%%%%%%%%%%%%%%%
\section{Introduction} 
\label{sec:introduction}
%%%%%%%%%%%%%%%%%%%%%%%%%%%%%%%%%%%%%%%%%%%%%%%%%%%%%%%%%%%%%%

Branching can greatly assist software development projects, 
particularly when developers need to both 
work on new features and maintain released code.
However, it can be difficult to remember 
all of the \texttt{git} commands for branching and merging.
This document gathers all commands in a single place.

This document gives my strategy for branching in a git repository. 
The strategy is based on the fine blog post by Vincent Driessen at\\
\url{http://nvie.com/posts/a-successful-git-branching-model/}
and other resources.
Adherence to this strategy will enable better development processes



\subsection{Subsection Heading Here}
Write your subsection text here.

% \begin{figure}
%     \centering
%     \includegraphics[width=3.0in]{myfigure}
%     \caption{Simulation Results}
%     \label{simulationfigure}
% \end{figure}

\section{Conclusion}
Write your conclusion here.

\end{document}
