\documentclass{article}

\usepackage{graphicx}
\graphicspath{ {figures/} }
\usepackage[hyphens]{url}

% Load last
\usepackage{hyperref}


\bibliographystyle{abbrv}

\begin{document}

%%%%%%%%%%%%%%%%%%%%%%%%%%%%%%%%%%%%%%%%%%%%%%%%%%%%%%%%%%%%%%
\title{Git Branching Strategy}
\author{Matthew Kuperus Heun}
%%%%%%%%%%%%%%%%%%%%%%%%%%%%%%%%%%%%%%%%%%%%%%%%%%%%%%%%%%%%%%

\maketitle


%%%%%%%%%%%%%%%%%%%%%%%%%%%%%%%%%%%%%%%%%%%%%%%%%%%%%%%%%%%%%%
\begin{abstract}
%%%%%%%%%%%%%%%%%%%%%%%%%%%%%%%%%%%%%%%%%%%%%%%%%%%%%%%%%%%%%%

This document gives my strategy for branching in a git repository. 
The strategy is based on the fine blog post by Vincent Driessen
and other resources.
Adherence to this strategy will enable better development processes
\end{abstract}


%%%%%%%%%%%%%%%%%%%%%%%%%%%%%%%%%%%%%%%%%%%%%%%%%%%%%%%%%%%%%%
\section{Introduction} 
\label{sec:introduction}
%%%%%%%%%%%%%%%%%%%%%%%%%%%%%%%%%%%%%%%%%%%%%%%%%%%%%%%%%%%%%%

The use of branches can greatly assist software development projects, 
particularly when developers need to both 
work on new features and maintain released code.

This document gives my model for branching in a git repository. 
The strategy is based on the fine blog post by Vincent Driessen~\cite{Driessen:2010}
and other resources~\cite{Onkelinx:2017, Rankin:2010}.
The model assumes the existence of \texttt{github} or a similar remote repository
and use of the RStudio IDE, 
although the model (if not the details) 
will apply to other IDEs that support branching. 

In addition, it can be difficult to remember 
all of the \texttt{git} commands for branching and merging.
This document gathers all commands in a single place
and discusses the commands in the context of the overall workflow.
 
Adherence to this strategy and use of these commands 
will improve my software development processes.


%%%%%%%%%%%%%%%%%%%%%%%%%%%%%%%%%%%%%%%%%%%%%%%%%%%%%%%%%%%%%%
\section{Workflow} 
\label{sec:workflow}
%%%%%%%%%%%%%%%%%%%%%%%%%%%%%%%%%%%%%%%%%%%%%%%%%%%%%%%%%%%%%%

The workflow presented here assumes familiarity 
with the Driessen branching model. 
All \texttt{git} commands can be entered 
at the command line of the developer's local computer
in the root directory of the repository.
Both Terminal.app (MacOS) and the Terminal window of RStudio
can be used for this purpose.


%++++++++++++++++++++++++++++++
\subsection{Setup} 
\label{sec:setup}
%++++++++++++++++++++++++++++++

Use the following workflow to establish a repository and 
create the master (release) and develop branches.

\begin{enumerate}

  \item Create a git repository

  \item Clone the repository, duplicating the \texttt{master} branch on the local computer:
  		\texttt{git clone git@github.com:MatthewHeun/<<RepositoryName>>.git}
  
  \item Create the \texttt{develop} branch: 
  		\texttt{git checkout -b develop master}

  \item If needed, click the refresh icon 
  		(\raisebox{-0.3\height}{\includegraphics[scale=0.5]{refresh}}) in the Git pane of RStudio to 
  		show that the project is now on the \texttt{develop} branch.
  
\end{enumerate}


%++++++++++++++++++++++++++++++
\subsection{Work on a feature} 
\label{sec:feature}
%++++++++++++++++++++++++++++++

\begin{enumerate}

  \item Create a feature branch 

  \item

\end{enumerate}





\section{Conclusion}

Write your conclusion here.




\bibliography{references}



\end{document}
